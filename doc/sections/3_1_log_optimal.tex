\subsection{The Log-optimal Portfolio}
Um dos primeiros resultados teóricos da informação relacionados ao investimento e à teoria dos jogos é a Aposta Proporcional (também chamada de Kelly Betting) \cite{kelly1}. Em seu artigo original, Kelly mostrou que essa teoria era sólida para cenários de jogo, com resultados binários (ganhar / perder). Escolher o que as apostas para fazer em um cenário de jogo pode ser visto como um caso específico do problema de seleção de portfólio mais geral. Mais tarde, demonstrou-se que essa teoria ainda é válida nos casos não binários mais amplos.

Como na Seção 2, aqui um investidor ainda procura maximizar seu retorno esperado sobre a riqueza investida. No entanto, os investidores geralmente não colocam investimentos e os removem no próximo período. Em vez disso, eles reinvestem seus ganhos, independentemente de o investimento ter pago ou não. Isso significa que o que está tentando maximizar é realmente a taxa de crescimento de seu investimento, que é a taxa em que sua riqueza cresce ao longo dos vários períodos. Kelly mostrou que a taxa de crescimento $ W(\bf{b}, F) $ de uma carteira de investimento $ \bf{b} $ dada uma distribuição de mercado $ F(\bf{x}) $ é:

\[W({\bf{b}},F) = E\left({{{\log }}{{\bf{b}}^t}{\bf{X}}}\right)\]

Quando o logaritmo de 2 é usado, o \emph{growth rate} de um portfólio é chamado \emph{doubling rate}, o que essencialmente nos diz o tempo que leva para um determinado portfólio duplicar em valor. Portanto, usando a definição de taxa de crescimento, o que um investidor deve fazer é encontrar o portfólio que maximiza essa taxa:

\[{W^*}(F) = \mathop {\max }\limits_{\bf{b}} W({\bf{b}},F)\]

Pode ser mostrado (sobre provas formais) que é igual a:

\[{W^*}(F) = \mathop {\max }\limits_{\bf{b}} E(\log \bf{b}^t \bf{X})\]

O que significa que, em vez de tentar maximizar o retorno esperado de suas riquezas, um investidor racional deve, em vez disso, maximizar o valor esperado do \emph{logaritmo} de suas riquezas. Isso também significa que eles devem investir (ou no caso de Kelly, fazer apostas) pesados pelo logaritmo de seus resultados esperados, não pelos próprios resultados.

As portifólios de \emph{log-optimal} mostraram ser astimamente ótimas, o que significa no longo prazo (quando o número de períodos de investimento tende ao infinito), eles alcançam a taxa de crescimento máxima. Eles também mostraram ser competitivos otimizados. Isso significa que um investidor que investe usando um portfólio \emph{log-optimal} pode sempre desempenhar o mesmo ou superar um investidor usando uma portifólio casualmente escolhida, mas nunca pode executar abaixo deles.