Na seção anterior, mostramos como, com o conhecimento da distribuição de probabilidade de um mercado, a estratégia de investimento ótima para qualquer investidor é um portfólio \emph{log-optimal}. No entanto, assumindo que todos os investidores são racionais e tomam decisões ótimas, e que todos os investidores tenham conhecimento das mesmas informações sobre o mercado, todos os participantes no mercado de ações fariam os mesmos investimentos e ninguém jamais teria uma vantagem.

Enfrentar o primeiro pressuposto teria que aprofundar a teoria dos jogos e a análise dos diferentes fatores que afetam a tomada de decisões de uma determinada pessoa, o que não será abrangido pelo escopo deste trabalho. A segunda hipótese, por outro lado, é claramente não razoável. Diferentes investidores têm diferentes graus de conhecimento sobre o mercado, então nosso modelo deve adequadamente ajustar esta observação. Podemos então voltar nossa atenção para o problema de, dada a informação pública "comum" sobre o mercado $ X $ e o conhecimento privilegiado de um investidor específico sobre o mercado $ Y $, medindo o potencial que este investidor pode acumular em sua riqueza abusando esse conhecimento extra.
