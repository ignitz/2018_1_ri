% This is "sig-alternate.tex" V2.1 April 2013
% This file should be compiled with V2.5 of "sig-alternate.cls" May 2012
%
% This example file demonstrates the use of the 'sig-alternate.cls'
% V2.5 LaTeX2e document class file. It is for those submitting
% articles to ACM Conference Proceedings WHO DO NOT WISH TO
% STRICTLY ADHERE TO THE SIGS (PUBS-BOARD-ENDORSED) STYLE.
% The 'sig-alternate.cls' file will produce a similar-looking,
% albeit, 'tighter' paper resulting in, invariably, fewer pages.
%
% ----------------------------------------------------------------------------------------------------------------
% This .tex file (and associated .cls V2.5) produces:
%       1) The Permission Statement
%       2) The Conference (location) Info information
%       3) The Copyright Line with ACM data
%       4) NO page numbers
%
% as against the acm_proc_article-sp.cls file which
% DOES NOT produce 1) thru' 3) above.
%
% Using 'sig-alternate.cls' you have control, however, from within
% the source .tex file, over both the CopyrightYear
% (defaulted to 200X) and the ACM Copyright Data
% (defaulted to X-XXXXX-XX-X/XX/XX).
% e.g.
% \CopyrightYear{2007} will cause 2007 to appear in the copyright line.
% \crdata{0-12345-67-8/90/12} will cause 0-12345-67-8/90/12 to appear in the copyright line.
%
% ---------------------------------------------------------------------------------------------------------------
% This .tex source is an example which *does* use
% the .bib file (from which the .bbl file % is produced).
% REMEMBER HOWEVER: After having produced the .bbl file,
% and prior to final submission, you *NEED* to 'insert'
% your .bbl file into your source .tex file so as to provide
% ONE 'self-contained' source file.
%
% ================= IF YOU HAVE QUESTIONS =======================
% Questions regarding the SIGS styles, SIGS policies and
% procedures, Conferences etc. should be sent to
% Adrienne Griscti (griscti@acm.org)
%
% Technical questions _only_ to
% Gerald Murray (murray@hq.acm.org)
% ===============================================================
%
% For tracking purposes - this is V2.0 - May 2012

\documentclass{sig-alternate-05-2015}
\usepackage[brazilian]{babel}
\usepackage[utf8]{inputenc}
\usepackage[T1]{fontenc}

\begin{document}

%Conference

%
% --- Author Metadata here ---
%\CopyrightYear{2007} % Allows default copyright year (20XX) to be over-ridden - IF NEED BE.
%\crdata{0-12345-67-8/90/01}  % Allows default copyright data (0-89791-88-6/97/05) to be over-ridden - IF NEED BE.
% --- End of Author Metadata ---

\title{Universal Portfolio Algorithm}
%
% You need the command \numberofauthors to handle the 'placement
% and alignment' of the authors beneath the title.
%
% For aesthetic reasons, we recommend 'three authors at a time'
% i.e. three 'name/affiliation blocks' be placed beneath the title.
%
% NOTE: You are NOT restricted in how many 'rows' of
% "name/affiliations" may appear. We just ask that you restrict
% the number of 'columns' to three.
%
% Because of the available 'opening page real-estate'
% we ask you to refrain from putting more than six authors
% (two rows with three columns) beneath the article title.
% More than six makes the first-page appear very cluttered indeed.
%
% Use the \alignauthor commands to handle the names
% and affiliations for an 'aesthetic maximum' of six authors.
% Add names, affiliations, addresses for
% the seventh etc. author(s) as the argument for the
% \additionalauthors command.
% These 'additional authors' will be output/set for you
% without further effort on your part as the last section in
% the body of your article BEFORE References or any Appendices.

\numberofauthors{4} %  in this sample file, there are a *total*
% of EIGHT authors. SIX appear on the 'first-page' (for formatting
% reasons) and the remaining two appear in the \additionalauthors section.
%
\author{
% You can go ahead and credit any number of authors here,
% e.g. one 'row of three' or two rows (consisting of one row of three
% and a second row of one, two or three).
%
% The command \alignauthor (no curly braces needed) should
% precede each author name, affiliation/snail-mail address and
% e-mail address. Additionally, tag each line of
% affiliation/address with \affaddr, and tag the
% e-mail address with \email.
%
% 1st. author
\author XYuri Niitsuma\\
       \affaddr{Universidade Federal de Minas Gerais}\\
       \email{yuriniitsuma@dcc.ufmg.br}
\and  % use '\and' if you need 'another row' of author names
}
% There's nothing stopping you putting the seventh, eighth, etc.
% author on the opening page (as the 'third row') but we ask,
% for aesthetic reasons that you place these 'additional authors'
% in the \additional authors block, viz.
\date{19 November 2017}
% Just remember to make sure that the TOTAL number of authors
% is the number that will appear on the first page PLUS the
% number that will appear in the \additionalauthors section.

\maketitle
\begin{abstract}
O campo da Teoria da Informação foi inicialmente criado sobre o estudo de canais de comunicação. Em que a maior motivação foi recriar a entrada original do canal, dado a possibilidade de erro na saída. Vários conceitos foram criados, desde a definição formal matemática, até métricas de mensuração. Mas esta ferramenta em outros campos não relacionados, mostrando o vasto escopo de aplicabilidade. Uma diferença entre a Teoria da Decisão é que analisa o ganho de um jogador enquanto a Teoria dos Jogos analisa de todos os envolvidos.

Iremos apresentar e analisar algumas aplicações em que neste trabalho vamos analisar o campo de investimento em ações utilizando um conceito simples de Teoria da Informação chamado Universal Portfolio. que tem relação com Teoria de Decisão na escolha da melhor proporção de investimento em cada conjunto de ações.

**Não encontrei uma tradução exata de \emph{portfolio}, que é um conjunto ações de investimento. Portanto, a palavra que será utilizada será a tradução literal \emph{portifólio}.

\end{abstract}

\section{Introdução} O trabalho consiste em criar um crawler que coletará
links (internos e externos) e criará um arquivo de dump contendo a url e a
página em html conforme está na especificação.

Alguns requisitos do crawler são:



\begin{itemize}
  \item{Formato do arquivo de dump:}

  |||<url>|<conteúdo>|||<url>|<conteúdo>|||...

  |||<url>|<conteúdo>|||


  \item{Código em C++}
  \item{Uso da biblioteca Chilkat é recomendado}
  \item{Baixar páginas no domínio *.br}
  \item{Coletar somente páginas HTML}
  \item{Coletar páginas com no máximo 2 MB}
  \item{Mínimo: 1 milhão de páginas}
  \item{Máximo: 5 milhões de páginas}
  \item{Comprimir o arquivo final no formato .tar}
\end{itemize}

Todos os requisitos foram atendidos com exceção do mínimo da
coleção de páginas que será explicado a seguir.


\newpage
% \section{Conclusão}

Neste trabalho utilizamos conhecimento base de teoria da informação e teoria de decisão para escolher as melhores estratégias de investimento em ações durante períodos de tempo.
Através de definições matemáticas é possível encontrar a melhor estratégia apesar do sistema ser estocástico e iterativo.

%\end{document}  % This is where a 'short' article might terminate
%
% The following two commands are all you need in the
% initial runs of your .tex file to
% produce the bibliography for the citations in your paper.
\bibliographystyle{abbrv}
\bibliography{sigproc}  % sigproc.bib is the name of the Bibliography in this case

% You must have a proper ".bib" file
%  and remember to run:
% latex bibtex latex latex
% to resolve all references
%
% ACM needs 'a single self-contained file'!
%
%APPENDICES are optional
%\balancecolumns
\end{document}
